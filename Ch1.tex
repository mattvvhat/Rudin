\documentclass{article}

\usepackage{amssymb,amsmath,amsfonts,amsthm}

\newcommand{\set}[2]{
	\{ #1 \mid #2 \}
}
\newcommand{\qeq}{\stackrel{?}{=}}

\begin{document}

Chapter 1.

\section{Introduction}

\subsection{Rational Numbers}

$
	\forall p \in \mathbb{Q}, \exists m, n \in \mathbb{Z}, \mid p = \frac{m}{n}
$

\subsection{Irrational Numbers}

$
	\exists p \in \mathbb{R}, s.t. \forall m, n \in \mathbb{Z}, p \neq \frac{m}{n}
$

\subsubsection{Example}

Suppose:
\begin{align*}
	p \in \mathbb{Q} \\
	m,n \in \mathbb{Z}, st. p =j \frac{m}{n} \\
	gcd(m, n) = 1 \\
\end{align*}

\begin{align*}
	p^2 & < 2 \\
	\frac{m^2}{n^2} & < 2 \\
	m^2 & < 2 n^2
\end{align*}

Then:
\begin{align*}
	q				& = p - \frac{p^2-2}{p+2} \\
	q				& = \frac{2p-2}{p+2} \\
	q^2			& = \frac{4p^2 - 8p - 4}{(p+2)^2} \\
	q^2 - 2	& = \frac{(4p^2 - 8p - 4) - 2 (p+2)^2}{(p+2)^2} \\
	q^2 - 2	& = \frac{(4p^2 - 8p + 4) - (2p^2 + 8p + 8)}{(p+2)^2} \\
	q^2 - 2	& = \frac{2p^2 - 16p - 4}{(p+2)^2}
\end{align*}

\section{Ordered Sets}

{\bf Definition.} Let S be a set. An \emph{order} on S is a relation, denoted by <, with the following two properties:

\begin{equation}
	If x,y \in S, \\
	Then x < y, x = y, or y < x
\end{equation}

\begin{align*}
	If: 		&	x, y, z \in S \\
	And:		& x < y, y < z \\
	Then:		&	x < z
\end{align*}

{\bf Definition.} An \emph{ordered set} is a set S in which an order is defined.

{\bf Definition.}
	Suppose S is an ordered set, and $ E \subset S $.
	If there $ \exists \beta \in S, st. \forall x \in E, x < \beta $, then E is \emph{bounded above} by $\beta$.
	
{\bf Definition.}
	Suppose $\mathbf{S}$ is an ordered set, $\mathbf{E} \subset \mathbf{S}$, and $\mathbf{E}$ is bounded above. Suppose there exists an $\alpha \in S$ with the following properties:
	\begin{gather*}
		\forall x \in \mathbf{E}, \quad x < \alpha \\
		If \quad \gamma < \alpha, \quad then \quad \gamma is not an upper bound of \mathbf{E}
	\end{gather*}
	
	Then:
		$$ \alpha = \sup \mathbf{E} $$
		
\subsection{Examples}
	(a) ***Ignored \\
	(b) If $\alpha = \sup \mathbf{E}$ exists, then $\alpha$ may or may not be in $\mathbf{E}$. \\
	(c) Let $\mathbf{E}$ consist of all numbers $\frac{1}{n}$, when n = 1\dots. Then, $\sup\mathbf{E} = 1$ and $\inf\mathbf{E} = 0$.
	
{\bf Definition.} An ordered set $\mathbf{S}$ is said to have the \emph{least-upperbound property} if the following is true: \\

\begin{enumerate}
	\item
		If $\mathbf{E} \subset \mathbf{S}$, $\mathbf{E}$ is not empty, and $\mathbf{E}$ is bounded above,
		then $\sup\mathbf{E} \in \mathbf{S}$
\end{enumerate}


{\bf Given.}
	$$ \mathbf{S} \text{is an ordered set with a least upper bound} $$
	$$ \alpha := x $$
	$$ \mathbf{B} \subset \mathbf{S} $$
	$$ \mathbf{B} \neq \emptyset $$
	$$ \mathbf{L} := \set{y \in \mathbf{S}}{y \leq x \quad \forall x \in \mathbf{B}} $$

{\bf Prove.}
	$$ \alpha = \sup\mathbf{L} $$
	$$ \alpha = \inf\mathbf{B} $$
	
{\bf Proof.}
	$$ \text{Since $\mathbf{L}$ is bounded below, } \mathbf{L} \neq \emptyset $$
	$$ \text{Every $\mathbf{B}$ is an upper-bound for $\mathbf{L}$} $$
	$$ \text{$\mathbf{L}$ is bounded above, because $\mathbf{B}$ is nonempty} $$
	$$ \sup\mathbf{L} \in \mathbf{S} \text{, because $\mathbf{S}$ has a least upperbound $\beta$, which is greater than everything in $\mathbf{S}$ and $\mathbf{L}$} $$
	$$ \text{If } \gamma < \alpha, \gamma \text{ is not an upper-bound of } \mathbf{L} \text{, and } \gamma \notin \mathbf{B} $$
	$$ \text{Thus } \alpha \leq x, \quad \forall x \in \mathbf{B} $$
	$$ \text{Thus } \alpha \in \mathbf{L} $$
	$$ \text{But } \alpha \notin \mathbf{B} $$
	$$ \text{Then this kinda becomes too confusing for me to follow on computer, because $\gamma$ comes out of nowhere} $$
	
\section{Fields}

{\bf Definition.}
	A \emph{field} is a set $\mathbf{F}$ with two operations, called \emph{addition} and \emph{multipleication}, which satisfy the field axioms:
	\begin{description}
		\item[\emph{closure addition}:]
			If $x \in \mathbf{F}$, then their sum $x + y \in \mathbf{BF}$
		\item[\emph{commutative addition}:]
			$x + y = y + x$
		\item[\emph{assosciative addition}:]
			$(x + y) + z = x + (y + z)$
		\item[\emph{identity addition}:]
			$\mathbf{F}$ contains an element $0$ s.t. $0 + x = x, \quad \forall x\in F$
		\item[\emph{Inverse addition}:]
			$\forall x \in \mathbf{F}, \quad \exists -x \in \mathbf{F}, \text{ s.t. } x + (-x) = 0$
		\item[\emph{Closure (multiplication)}:]
			If $x \in \mathbf{F}$, then their product $x \cdot y \in \mathbf{BF}$
		\item[\emph{Commutative (multiplication)}:]
			$x \cdot y = y \cdot x$
		\item[\emph{Assosciative (multiplication)}:]
			$(x \cdot y) \cdot z = x \cdot (y \cdot z)$
		\item[\emph{Identity (multiplication)}:]
			$\mathbf{F}$ contains an element $1$ s.t. $1 \cdot x = x, \quad \forall x\in F$
		\item[\emph{Inverse (multiplication)}:]
			$\forall x \in \mathbf{F\setminus\{0\}}, \quad \exists x^{-1} \in \mathbf{F}, \text{ s.t. } x \cdot (x^{-1}) = 1$
		\item[\emph{Distribution}:]
			$x \cdot (y + z) = x \cdot y + x \cdot z, \quad \forall x, y, z \in \mathbf{F}$
			
	\end{description}


\subsection{Proposition}

{\noindent\bf Proposition}

\begin{enumerate}
	\item If $x + y = x + z$, then $y = z$
	\item If $x + y = x$, then $y = 0$
	\item If $x + y = 0$, then $y = -x$
	\item $-(-x) = x$
\end{enumerate}

{\bf Proof.}

(a)
\begin{align*}
	y		& = 0 + y					\\
			& = (x + (-x)) + y\\
			& = -x + (x + y)	\\
			& = -x + (x + z)	\\
			& = (-x + x) + z	\\
			& = 0 + z					\\
			& = z
\end{align*}
(b)
\begin{align*}
	x + y		& = x				\\
	x + y		& = x + 0		\\
	\text{Thus } y = 0 \text{ by the previous proof}
\end{align*}

(c)
\begin{align*}
	\text{Skipped.}
\end{align*}

(d)
\begin{align*}
	-(-x) & \qeq \dots		\\
	y 		& := -x					\\
	-y		& = 0 - y				\\ 
	-y		& = (x - x) - y \\
				& = (x + y) - y	\\
				& = x + (y - y)	\\
				& = x + 0				\\
				& = x						\\
	-(-x)	& = x		
\end{align*}


\pagebreak

{\noindent\bf Definition.}
	An \emph{Ordered Field} is a \emph{Field} $\mathbf{F}$, which is also an \emph{ordered set}, such that:
	\begin{itemize}
		\item $x + y < x + z$, if $x, y, z \in \mathbf{F}$ and $y < z$
		\item $x \cdot y > 0$ if $x, y \in \mathbf{F}$, $x > 0$, and $y > 0$
	\end{itemize}


\subsection{The Real Field}

{\noindent\bf Theorem.}
	There exists an ordered field $\mathbf{R}$ which has the least upperbound property.
	$\mathbb{Q} \subset \mathbb{R}$
	
{\noindent\bf Theorem.}
	\begin{itemize}
		\item If $x, y \in \mathbb{R}$ and $x > 0$, then $\exists n \in \mathbb{Z}$ st. $nx > y$
		\item If $x, y \in \mathbb{R}$ and $x < y$, then $\exists p \in \mathbb{Q}$ st. $x < p < y$
	\end{itemize}

{\noindent\bf Proof.}
	(a)
	$$ A := \set{nx}{n \in \mathbb{Z}} $$
	$$ \text{Assume to the contrary that } \forall n \in \mathbb{Z}, nx \le y $$
	$$ \text{Then } y \text{ is an upperbound of } \mathbf{A} $$
	$$ \text{So define } \alpha := \sup A $$
	$$ \alpha - x < \alpha \quad\because x > 0 $$
	$$ \exists m \in \mathbb{Z} \text{, st. } \alpha - x < mx \quad\because a - x \text{ is not an upperbound of } \mathbf{A} $$
	$$ \text{But then } \alpha < (m+1)x \in A \text{, which impossible because $\alpha$ is an upperbound for $\mathbf{A}$}$$
	$$ \qed \text{(a) is true, aka the contrary is absurd} $$










\end{document}