\documentclass{article}

\usepackage{amssymb,amsmath,amsfonts,amsthm}
% Set builder notation
\newcommand{\set}[2]{
	\{\ #1\ \mid\ #2\ \}
}

% ?=
\newcommand{\qeq}{\stackrel{?}{=}}

% def=
\newcommand{\defeq}{\stackrel{\text{def}}{=}}

% Definition block with label
\newcommand{\DEFINITION}[1]{
  \label{def-#1}
  {\noindent \bf Definition #1}
}

% Theorem block with label
\newcommand{\THEOREM}[1]{
  \label{theorem-#1}
  {\noindent \bf #1 Theorem}
}

% Generator: <a>
\newcommand{\gen}[1]{
  \langle #1 \rangle
}

% Modulo: (mod 3)
\newcommand{\modu}[1]{
  \ (\textrm{mod}\ #1)
}


\begin{document}

\section{Rings}

% RING HOMOMORPHISMS
% ...
% ...
\subsection{Ring Homomorphisms}

$\phi$ is a \emph{ring homomorphism}, if:
\begin{itemize}
  \item $\phi(a+b)=\phi(a)\oplus\phi(b)$
  \item $\phi(a \cdot b)=\phi(a)\otimes\phi(b)$
\end{itemize}

% FEATURES OF RING HOMOMORPHISMS
% ...
% ...
\subsection{Features of Ring Homomorphisms}


\begin{enumerate}
  \item
    The \textbf{kernel} of a ring homomorphism is the set:\\
    $\ker \phi \defeq \set{a\in\mathbf{R}}{\phi(a)=0'}$ \\
    Note that $\ker\phi: \mathbf{R} \mapsto \mathbf{R'}$ is a subring of
    $\mathbf{R}$
  \item
    The image of $\mathbf{R}$, $\phi(\mathbf(R)$ is a subring of $\mathbf{R'}$
  \item
    The image of $0_+\in\mathbf{R}$ is $0_+'\in\mathbf{R'}$.\\
    Note that this means: $\phi(-a)=-\phi(a)$

\end{enumerate}


\subsection{Extensions of Rings}

{\noindent}Rings are sets $\mathbf{R}$ where:
\begin{itemize}
  \item $(\mathbf{R}, +)$ is an Abelian group
  \item $(\mathbf{R}, \cdot)$ is a semigroup (multiplication is associative)
  \item
    The \emph{distributive law} holds:
    $a \cdot (b + c) = (a + b) \cdot c = a \cdot c + a \cdot b$
\end{itemize}

{\noindent}Types of rings, and elemts:
\begin{itemize}
  \item
    A {\bf Ring with Unity} is a a ring where $(R, \cdot)$ is a
    monoid (closed, associative, identity)
  \item
    A {\bf Unit} is a \emph{ring element} that has a multiplicative
    inverse.
  \item
    A {\bf Division Ring} is a ring where every nonzero element has a
    \emph{multiplicative inverse}.
  \item
    An {\bf Integral Domain} is a commutative ring, where:
    $a \cdot b = 0 \iff a=0 \lor b=0$
  \item
    A {\bf Field} is a commutative \emph{division ring}.
\end{itemize}

\end{document}
