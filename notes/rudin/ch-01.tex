% Review + Learning Latex
%
%

\documentclass{article}

\usepackage{amssymb,amsmath,amsfonts,amsthm}

\newcommand{\set}[2]{
	\{ #1 \mid #2 \}
}
\newcommand{\qeq}{\stackrel{?}{=}}
\newcommand{\defeq}{\stackrel{\text{def}}{=}}


\newcommand{\DEFINITION}[1]{
  \label{def-#1}
  {\noindent \bf Definition #1}
}

\newcommand{\THEOREM}[1]{
  \label{theorem-#1}
  {\noindent \bf #1 Theorem}
}

\begin{document}

Chapter 1.

\section{Introduction}

\subsection{Rational Numbers}

$
	\forall p \in \mathbb{Q}, \exists m, n \in \mathbb{Z}, \mid p = \frac{m}{n}
$

\subsection{Irrational Numbers}

$
	\exists p \in \mathbb{R}, s.t. \forall m, n \in \mathbb{Z}, p \neq \frac{m}{n}
$

\subsubsection{Example}

Suppose:
\begin{align*}
	p \in \mathbb{Q} \\
	m,n \in \mathbb{Z}, st. p =j \frac{m}{n} \\
	gcd(m, n) = 1 \\
\end{align*}

\begin{align*}
	p^2 & < 2 \\
	\frac{m^2}{n^2} & < 2 \\
	m^2 & < 2 n^2
\end{align*}

Then:
\begin{align*}
	q				& = p - \frac{p^2-2}{p+2} \\
	q				& = \frac{2p-2}{p+2} \\
	q^2			& = \frac{4p^2 - 8p - 4}{(p+2)^2} \\
	q^2 - 2	& = \frac{(4p^2 - 8p - 4) - 2 (p+2)^2}{(p+2)^2} \\
	q^2 - 2	& = \frac{(4p^2 - 8p + 4) - (2p^2 + 8p + 8)}{(p+2)^2} \\
	q^2 - 2	& = \frac{2p^2 - 16p - 4}{(p+2)^2}
\end{align*}

\section{Ordered Sets}

{\bf Definition.} Let S be a set. An \emph{order} on S is a relation, denoted by <, with the following two properties:

\begin{equation}
	If x,y \in S, \\
	Then x < y, x = y, or y < x
\end{equation}

\begin{align*}
	If: 		&	x, y, z \in S \\
	And:		& x < y, y < z \\
	Then:		&	x < z
\end{align*}

{\bf Definition.} An \emph{ordered set} is a set S in which an order is defined.

{\bf Definition.}
	Suppose S is an ordered set, and $ E \subset S $.
	If there $ \exists \beta \in S, st. \forall x \in E, x < \beta $, then E is \emph{bounded above} by $\beta$.
	
{\bf Definition.}
	Suppose $\mathbf{S}$ is an ordered set, $\mathbf{E} \subset \mathbf{S}$, and $\mathbf{E}$ is bounded above. Suppose there exists an $\alpha \in S$ with the following properties:
	\begin{gather*}
		\forall x \in \mathbf{E}, \quad x < \alpha \\
		If \quad \gamma < \alpha, \quad then \quad \gamma is not an upper bound of \mathbf{E}
	\end{gather*}
	
	Then:
		$$ \alpha = \sup \mathbf{E} $$
		
\subsection{Examples}
	(a) ***Ignored \\
	(b) If $\alpha = \sup \mathbf{E}$ exists, then $\alpha$ may or may not be in $\mathbf{E}$. \\
	(c) Let $\mathbf{E}$ consist of all numbers $\frac{1}{n}$, when n = 1\dots. Then, $\sup\mathbf{E} = 1$ and $\inf\mathbf{E} = 0$.
	
{\bf Definition.} An ordered set $\mathbf{S}$ is said to have the \emph{least-upperbound property} if the following is true: \\

\begin{enumerate}
	\item
		If $\mathbf{E} \subset \mathbf{S}$, $\mathbf{E}$ is not empty, and $\mathbf{E}$ is bounded above,
		then $\sup\mathbf{E} \in \mathbf{S}$
\end{enumerate}


{\bf Given.}
	$$ \mathbf{S} \text{is an ordered set with a least upper bound} $$
	$$ \alpha := x $$
	$$ \mathbf{B} \subset \mathbf{S} $$
	$$ \mathbf{B} \neq \emptyset $$
	$$ \mathbf{L} := \set{y \in \mathbf{S}}{y \leq x \quad \forall x \in \mathbf{B}} $$

{\bf Prove.}
	$$ \alpha = \sup\mathbf{L} $$
	$$ \alpha = \inf\mathbf{B} $$
	
{\bf Proof.}
	$$ \text{Since $\mathbf{L}$ is bounded below, } \mathbf{L} \neq \emptyset $$
	$$ \text{Every $\mathbf{B}$ is an upper-bound for $\mathbf{L}$} $$
	$$ \text{$\mathbf{L}$ is bounded above, because $\mathbf{B}$ is nonempty} $$
	$$ \sup\mathbf{L} \in \mathbf{S} \text{, because $\mathbf{S}$ has a least upperbound $\beta$, which is greater than everything in $\mathbf{S}$ and $\mathbf{L}$} $$
	$$ \text{If } \gamma < \alpha, \gamma \text{ is not an upper-bound of } \mathbf{L} \text{, and } \gamma \notin \mathbf{B} $$
	$$ \text{Thus } \alpha \leq x, \quad \forall x \in \mathbf{B} $$
	$$ \text{Thus } \alpha \in \mathbf{L} $$
	$$ \text{But } \alpha \notin \mathbf{B} $$
	$$ \text{Then this kinda becomes too confusing for me to follow on computer, because $\gamma$ comes out of nowhere} $$
	
\section{Fields}

{\bf Definition.}
	A \emph{field} is a set $\mathbf{F}$ with two operations, called \emph{addition} and \emph{multipleication}, which satisfy the field axioms:
	\begin{description}
		\item[\emph{closure addition}:]
			If $x \in \mathbf{F}$, then their sum $x + y \in \mathbf{BF}$
		\item[\emph{commutative addition}:]
			$x + y = y + x$
		\item[\emph{assosciative addition}:]
			$(x + y) + z = x + (y + z)$
		\item[\emph{identity addition}:]
			$\mathbf{F}$ contains an element $0$ s.t. $0 + x = x, \quad \forall x\in F$
		\item[\emph{Inverse addition}:]
			$\forall x \in \mathbf{F}, \quad \exists -x \in \mathbf{F}, \text{ s.t. } x + (-x) = 0$
		\item[\emph{Closure (multiplication)}:]
			If $x \in \mathbf{F}$, then their product $x \cdot y \in \mathbf{BF}$
		\item[\emph{Commutative (multiplication)}:]
			$x \cdot y = y \cdot x$
		\item[\emph{Assosciative (multiplication)}:]
			$(x \cdot y) \cdot z = x \cdot (y \cdot z)$
		\item[\emph{Identity (multiplication)}:]
			$\mathbf{F}$ contains an element $1$ s.t. $1 \cdot x = x, \quad \forall x\in F$
		\item[\emph{Inverse (multiplication)}:]
			$\forall x \in \mathbf{F\setminus\{0\}}, \quad \exists x^{-1} \in \mathbf{F}, \text{ s.t. } x \cdot (x^{-1}) = 1$
		\item[\emph{Distribution}:]
			$x \cdot (y + z) = x \cdot y + x \cdot z, \quad \forall x, y, z \in \mathbf{F}$
			
	\end{description}


\subsection{Proposition}

{\noindent\bf Proposition}

\begin{enumerate}
	\item If $x + y = x + z$, then $y = z$
	\item If $x + y = x$, then $y = 0$
	\item If $x + y = 0$, then $y = -x$
	\item $-(-x) = x$
\end{enumerate}

{\bf Proof.}

(a)
\begin{align*}
	y		& = 0 + y					\\
			& = (x + (-x)) + y\\
			& = -x + (x + y)	\\
			& = -x + (x + z)	\\
			& = (-x + x) + z	\\
			& = 0 + z					\\
			& = z
\end{align*}
(b)
\begin{align*}
	x + y		& = x				\\
	x + y		& = x + 0		\\
	\text{Thus } y = 0 \text{ by the previous proof}
\end{align*}

(c)
\begin{align*}
	\text{Skipped.}
\end{align*}

(d)
\begin{align*}
	-(-x) & \qeq \dots		\\
	y 		& := -x					\\
	-y		& = 0 - y				\\ 
	-y		& = (x - x) - y \\
				& = (x + y) - y	\\
				& = x + (y - y)	\\
				& = x + 0				\\
				& = x						\\
	-(-x)	& = x		
\end{align*}


\pagebreak

{\noindent\bf Definition.}
	An \emph{Ordered Field} is a \emph{Field} $\mathbf{F}$, which is also an \emph{ordered set}, such that:
	\begin{itemize}
		\item $x + y < x + z$, if $x, y, z \in \mathbf{F}$ and $y < z$
		\item $x \cdot y > 0$ if $x, y \in \mathbf{F}$, $x > 0$, and $y > 0$
	\end{itemize}


\subsection{The Real Field}

{\noindent\bf Theorem.}
	There exists an ordered field $\mathbf{R}$ which has the least upperbound property.
	$\mathbb{Q} \subset \mathbb{R}$
	
{\noindent\bf Theorem.}
	\begin{itemize}
		\item If $x, y \in \mathbb{R}$ and $x > 0$, then $\exists n \in \mathbb{Z}$ st. $nx > y$
		\item If $x, y \in \mathbb{R}$ and $x < y$, then $\exists p \in \mathbb{Q}$ st. $x < p < y$
	\end{itemize}

{\noindent\bf Proof.}
	(a)
	$$ A := \set{nx}{n \in \mathbb{Z}} $$
	$$ \text{Assume to the contrary that } \forall n \in \mathbb{Z}, nx \le y $$
	$$ \text{Then } y \text{ is an upperbound of } \mathbf{A} $$
	$$ \text{So define } \alpha := \sup A $$
	$$ \alpha - x < \alpha \quad\because x > 0 $$
	$$ \exists m \in \mathbb{Z} \text{, st. } \alpha - x < mx \quad\because a - x \text{ is not an upperbound of } \mathbf{A} $$
	$$ \text{But then } \alpha < (m+1)x \in A \text{, which impossible because $\alpha$ is an upperbound for $\mathbf{A}$}$$
	$$ \therefore \text{(a) is true, aka the contrary is absurd} $$
	
	(b)
	\begin{align*}
		\text{\bf Given.}	\\
			& x, y \in \mathbb{R} \land x < y			\\
		\text{\bf Show.}	\\
			& \exists p \in \mathbb{Q}: x < p < y \\
		\text{\bf Proof.}	\\
			& y - x > 0															& \because x < y				\\
			& \exists n \in \mathbb{Z}: n \cdot (y-x) > 1 & \because proof (a)		\\
			& \text{Let $m_1$ be the integer st. } nx < m_1									\\
			& \text{Let $m_2$ be the integer st. } -nx < m_2								\\
			& -m_2 < nx < m_1 \\
			& \exists m \in \mathbb{Z}: -m_2 \le m \le m_1 & \because \mathbb{Z}\text{is ordered} \\
			& m - 1 \le nx < m \\			
			& nx < m \le nx + 1 < ny & (!!!) \\
			& x < \frac{m}{n} < y & \because n > 0 \\
			& \text{Let } p = \frac{m}{n} \\
			& x < p < y \\
			& \qed... ask about me
	\end{align*}



  \subsection{}


  % Theorem
  %
  %
  {\bf 1.21 Theorem}

  \begin{align*}
    \text{\bf Prove.} \\
      & \forall x > 0 \in \mathbb{R}, \forall n > 0 \in \mathbb{Z}, \exists! y > 0 : y^n = x \\
    \text{\bf Note.} \\
      & \text{There is at most one $y > 0 : y^n = x$, because } \\
      & 0 < y_1 < y_2 \implies y^n < y^m \\
    \text{\bf Outline.} \\
      & \text{1. Okay} \\
      & \text{2. Okay} \\
    \text{\bf Proof.} \\
      & \mathbf{A} := \set{y \in \mathbb{R}}{ y > 0 \land y^n < x } \\
      & \text{Let } \alpha = \sup\mathbf{A} & \because \text{$\mathbf{A}$ is bound by $y$, there is a supremum} \\
      & \text{Assume to the contrary that $y^n < x$} & \text{Note: }\\
  \end{align*}

  Fuck



  \pagebreak

  \section{Extended Real Number Line}

  {\noindent \bf Definition 1.23}\label{def-1-23} The extended real reals $\defeq \mathbb{R} \cup { -\infty, +\infty } $ 

  {\noindent} $ \forall x \in \mathbb{R}, -\infty < x < \infty $

  Note that the extended-reals do not form a field, however the following is "customary":

  \begin{itemize}
    \item
      \begin{itemize}
        \item $ x + \infty = +\infty $
        \item $ x - \infty = -\infty $
        \item $ \frac{x}{+\infty} = \frac{x}{-\infty} = 0 $
      \end{itemize}
    \item
      \begin{itemize}
        \item If $x > 0$, then $x \cdot (+\infty) = +\infty$
        \item If $x > 0$, then $x \cdot (-\infty) = -\infty$
      \end{itemize}
    \item
      \begin{itemize}
        \item If $x < 0$, then $x \cdot (+\infty) = -\infty$
        \item If $x < 0$, then $x \cdot (-\infty) = +\infty$
      \end{itemize}
  \end{itemize}

  \section{The Complex Field $\mathbb{C}$}

  \DEFINITION{1.24} A \emph{complex} number is an ordered pair $(a, b)$ with $a, b \in \mathbb{R}$ 

  We define, $ \forall x, y \in \mathbb{C} $:
  \begin{itemize}
    \item $ x + y = (a + c, b + d) $
    \item $ x y = (ac - bd, ad + bc) $
    \item $ 0 = (0, 0) $
    \item $ I = (1, 0) $
  \end{itemize}

  \THEOREM{1.25} $\mathbb{C}$ is a field

  \begin{align*}
    {\bf Prove.} \\
      & \mathbb{C} \text{is a field} \\
    {\bf Given.} \\
      & x = (a, b), y = (c, d), z = (e, f)
    {\bf Proof.} \\
      & \text{(A1)} x + 0 = (a, b) + (0, 0) = (a + 0, b + 0) = (a, b) \\
      & \text{(A2)} x + y = (a, b) + (c, d) = (a + c, b + d) = (c + a, d + b) = y + x \\
      & \text{(A3)} (x + y) + z = (a + c, b + d) + (e, f) = (a + c + e, b + d + f) = (a, b) + (c + e, d + f) = x + (y+z)
      & \text{\emph{Skipping a few...}}
      & \text{M2} xy = (ac - bd, ad + bc) = (ca - db, da + cd) = yx
  \end{align*}

  \pagebreak

  \DEFINITION{1.30} If $a, b \in \mathbb{R}$ and $z = a + bi$, then the complex number $\overline{z} = a - bi$

  \THEOREM{1.31}
  \begin{itemize}
    \item $ \overline{z + w} = \overline{z} + \overline{w} $
    \item $ \overline{zw} = \overline{z} \cdot \overline{w} $
    \item $ z + \overline{z} = 2 \text{Re(z)}, z - \overline{z} = 2 i \text{Im(z)} $
    \item $ z \overline{z} \in \mathbb{R}^{+} \text{unless} z = 0 $
  \end{itemize}

  \DEFINITION{1.32} $ |z| = (z\overline{z})^{\frac{1}{2}} $

  \THEOREM{1.33} $z, w \in \mathbb{C} $
  \begin{itemize}
    \item $ |z| \geq 0; |z| = 0 \iff z = 0 $
    \item $ |\overline{z}| = |z| $
    \item $ |zw| = |z||w| $
    \item $ |\text{Re(z)}| \leq |z| $
    \item $ |z + w| \leq |z| + |w| $
  \end{itemize}

  Note that 1, 2, 3, 4 are \emph{really} easy. So here's 5:

  \begin{align*}
    {\bf Prove.} \\
      & |z + w| \leq |z| + |w| \\
    {\bf Given.} \\
      & z, w \in \mathbb{C} \\
    {\bf Proof.} \\
      & |z+w|^2 & = (z+w)(\overline{z+w}) = z\overline{z} + z\overline{w} + \overline{z}w + w\overline{w} \\
      & & = |z|^2 + 2\text{Re($z\overline{w}$)} + |w|^2 & \because |z|^2 = z\overline{z} \land \overline{z\overline{w}} = \overline{z}w \\
      & & \leq |z|^2 + 2\text{Re($z$)} + |w|^2 \\
      & & = |z|^2 2 |z| |w| + |w|^2 = (|z| + |w|)^2 \\
      & & \therefore |z+w| \leq |z| + |w| \because |z| + |w| \in \mathbb{R} > 0
  \end{align*}

  \THEOREM{1.35} If $ a_1, ..., a_n and b_1, ..., b_n \in \mathbb{C} $, then: \\
  {\indent} $ |\sum_{j=1}^n a_j\overline{b_j}|^2 \leq \sum_{j=1}^n |a_j|^2 \sum_{j=1}^n |b_j|^2$

  \pagebreak

  \begin{align*}
    {\bf Proof.}  \\
      & ...       \\
    {\bf Given.}  \\
      & A = \sum{a_j}^2 \\
      & B = \sum{b_j}^2 \\
      & C = \sum{a_j\overline{b_j}} \\
    {\bf Prove.}  \\
      \sum{|B a_j - C b_j|}^2 & = \sum{(B a_j - C b_j)(B\overline{a_j} - \overline{C b_j})} \\
      & = B^2 \sum{|a_j}^2 - B \overline{C} \sum{a_j\overline{b_j}} - B C \sum{\overline{a_j} b_j} + |C|^2 \sum{|b_j}^2 \\
      & = B^2 - B |C|^2 \\
      & = B (AB - |C|^2)
      & B (AB - |C|^2) > 0  & \because B>0  \\
      & AB - |C|^2 > 0      & \because B>0  \\
      & \therefore AB > |C|^2               \\
      & |C|^2 = |\sum{a_j\overline{b_j}}|^2 \leq \sum{|a_j|}^2\sum{|b_j|}^2 - AB \qed \\
  \end{align*}

  \pagebreak
  \section{Euclidean Spaces}

  \DEFINITION{1.36} $\mathbb{R}^k$ is the set of all ordered tuples, such that:
  $ \vec{x} = (x_1, x_2, \dots, x_k) : \forall k > 0, x_k \in \mathbb{R} $

  Vectors form a field! With $+$ and $*$

  In addition to field operations, vector-spaces contain \emph{scalar multiplication}, \emph{inner product}, and a \emph{norm}:

  \begin{itemize}
    \item $ \alpha \vec{x} = (\alpha x_1, \dots, \alpha x_k) $
    \item $ \vec{x} \cdot \vec{y} = \sum_{i=1}^k{x_i y_i}$
    \item $ |\vec{x}| = (\vec{x} \cdot \vec{x})^\frac{1}{2} $
  \end{itemize}

  {\bf Proof.} Schwartz Inequality \\
  
  $|x+y| \leq |x| + |y|, \forall x, y, z \in \mathbb{R}^k$

  \begin{align*}
    |x + y|^2 & =     (x+y) \cdot (x+y) \\
              & =     x \cdot x + 2 x \cdot y + y \cdot y \\
              & \leq  |x|^2 + 2|x||y| + |y|^2 \because x \cdot y \leq |x||y| \\
              & = (|x| + |y|)^2
  \end{align*}

\end{document}
