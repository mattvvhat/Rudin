\documentclass{article}
\usepackage{amssymb,amsmath,amsfonts,amsthm}
% Set builder notation
\newcommand{\set}[2]{
	\{\ #1\ \mid\ #2\ \}
}

% ?=
\newcommand{\qeq}{\stackrel{?}{=}}

% def=
\newcommand{\defeq}{\stackrel{\text{def}}{=}}

% Definition block with label
\newcommand{\DEFINITION}[1]{
  \label{def-#1}
  {\noindent \bf Definition #1}
}

% Theorem block with label
\newcommand{\THEOREM}[1]{
  \label{theorem-#1}
  {\noindent \bf #1 Theorem}
}

% Generator: <a>
\newcommand{\gen}[1]{
  \langle #1 \rangle
}

% Modulo: (mod 3)
\newcommand{\modu}[1]{
  \ (\textrm{mod}\ #1)
}

\begin{document}

\section{Number Theory and Abstract Algebra}

This takes up roughly 15\% of the test, but is partially involved in other
questions. This is a common mix-in, to make a problem harder.

\section{Divisibility}

\subsection{Quick Rules for Factoring}

\begin{itemize}
  \item by 2, iff the last digit is divisible by 2
  \item by 3, iff the sum of the digits is divisible by 3
  \item by 4, iff the last \emph{two} digitis is divisible by 4
  \item by 5, iff the last digit is 0 or 5
  \item by 8, iff the last \emph{three} digits are divisible by 8
  \item by 9, iff the sum of the digits is divisible by 9
\end{itemize}

\subsection{Division Algorithm}

If $a, b \in \mathbb{Z}^+$, then $\exists q, r \in \mathbb{Z} : b = qa + r$


\subsection{Primes}

$ \forall \in \mathbb{Z}^+, \exists \text{prime} p : k < p < 2k $

$ \sum{k=1}^n{\frac{1}{p_k}} $ divierges where $p_k$ is the k-th prime

\subsection{GCD and LCD}

Greatest-common-division and least-common-denominator come up in group size and
and various algorithmic problems.

\pagebreak
\subsubsection{GCD}

...

For any integers $a, b$, we can write:
  \begin{align*}
    a   & = (p_1)^{a_1} (p_2)^{a_2} ... (p_k)^{a_k}     \\
    b   & = (p_1)^{b_1} (p_2)^{b_2} ... (p_k)^{b_k}     \\
    m_i & \defeq min(a_i, b_i)                          \\
    M_i & \defeq max(a_i, b_i)                          \\
    gcd(a, b) & \defeq (p_1)^{m_1} (p_2)^{m_2} (\dots) (p_k)^{m_k} \\
    lcM(a, b) & \defeq (p_1)^{M_1} (p_2)^{M_2} (\dots) (p_k)^{M_k} \\
  \end{align*}

As a result:
  $$ gcd(a, b) \cdot lcm(a, b) = a \cdot b $$

\subsubsection{Euclidean Algorithm}

Algorithmically determine the greatest common divisor between two numbers. This
divides the larger number by the smaller and continually checks whether it did
it evenly. When $r = 0$ we know $ r | b_k, b_{k-1}, b_{k-2}, ..., b_0 $ and that
it must be the largest such number that does it.

\begin{verbatim}
  def gcd (a, b):
    if b < a: (a, b) = (b, a)

    r = b % a

    while r != 0:
      # Note: b = a * m + r
      b = a
      a = r
      m = b // a
      r = b % a

    return a
\end{verbatim}

\subsubsection{Diophantine Equation}

\emph{Starting here tomorrow}

\end{document}
