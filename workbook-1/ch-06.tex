\documentclass{article}
\usepackage{amssymb,amsmath,amsfonts,amsthm}
% Set builder notation
\newcommand{\set}[2]{
	\{\ #1\ \mid\ #2\ \}
}

% ?=
\newcommand{\qeq}{\stackrel{?}{=}}

% def=
\newcommand{\defeq}{\stackrel{\text{def}}{=}}

% Definition block with label
\newcommand{\DEFINITION}[1]{
  \label{def-#1}
  {\noindent \bf Definition #1}
}

% Theorem block with label
\newcommand{\THEOREM}[1]{
  \label{theorem-#1}
  {\noindent \bf #1 Theorem}
}

% Generator: <a>
\newcommand{\gen}[1]{
  \langle #1 \rangle
}

% Modulo: (mod 3)
\newcommand{\modu}[1]{
  \ (\textrm{mod}\ #1)
}


\newcommand{\congr}[3]{
  #1 \equiv #2\ (\textrm{mod}\ #3)
}
\begin{document}

\section{Number Theory and Abstract Algebra}

This takes up roughly 15\% of the test, but is partially involved in other
questions. This is a common mix-in, to make a problem harder.

\section{Divisibility}

\subsection{Quick Rules for Factoring}

\begin{itemize}
  \item by 2, iff the last digit is divisible by 2
  \item by 3, iff the sum of the digits is divisible by 3
  \item by 4, iff the last \emph{two} digitis is divisible by 4
  \item by 5, iff the last digit is 0 or 5
  \item by 8, iff the last \emph{three} digits are divisible by 8
  \item by 9, iff the sum of the digits is divisible by 9
\end{itemize}

\subsection{Division Algorithm}

If $a, b \in \mathbb{Z}^+$, then $\exists q, r \in \mathbb{Z} : b = qa + r$


\subsection{Primes}

$ \forall \in \mathbb{Z}^+, \exists \text{prime} p : k < p < 2k $

$ \sum{k=1}^n{\frac{1}{p_k}} $ divierges where $p_k$ is the k-th prime

\subsection{GCD and LCD}

Greatest-common-division and least-common-denominator come up in group size and
and various algorithmic problems.

\pagebreak
\subsubsection{GCD}

...

For any integers $a, b$, we can write:
  \begin{align*}
    a   & = (p_1)^{a_1} (p_2)^{a_2} ... (p_k)^{a_k}     \\
    b   & = (p_1)^{b_1} (p_2)^{b_2} ... (p_k)^{b_k}     \\
    m_i & \defeq min(a_i, b_i)                          \\
    M_i & \defeq max(a_i, b_i)                          \\
    gcd(a, b) & \defeq (p_1)^{m_1} (p_2)^{m_2} (\dots) (p_k)^{m_k} \\
    lcM(a, b) & \defeq (p_1)^{M_1} (p_2)^{M_2} (\dots) (p_k)^{M_k} \\
  \end{align*}

As a result:
  $$ gcd(a, b) \cdot lcm(a, b) = a \cdot b $$

\subsubsection{Euclidean Algorithm}

Algorithmically determine the greatest common divisor between two numbers. This
divides the larger number by the smaller and continually checks whether it did
it evenly. When $r = 0$ we know $ r | b_k, b_{k-1}, b_{k-2}, ..., b_0 $ and that
it must be the largest such number that does it.

\begin{verbatim}
  def gcd (a, b):
    if b < a: (a, b) = (b, a)

    r = b % a

    while r != 0:
      # Note: b = a * m + r
      b = a
      a = r
      m = b // a
      r = b % a

    return a
\end{verbatim}

\subsubsection{Diophantine Equation $ax +by = c$}

The Simple Linear Diophantine Equations $ax + by = c$ has solutions of the form:\\
  $ x = x_1 + \frac{b}{d} t $\\
  $ y = y_1 - \frac{a}{d} t $

  Finding $x_1$ and $y_1$ is the only real hard part. We know that we can \emph{always} find solutions of the form: \\
  $ a x_0 + b y_0 = d $ where $d = gcd(a, b)$

  After solving this equation, we multiply both sides of the equation by $\frac{c}{d}$ to get:
  $ a \frac{c}{d} x_0 + b \frac{c}{d} y_0 = d \frac{c}{d} $

  Define: \\
  $ x_1 = \frac{c}{d} x_0 $ \\
  $ y_1 = \frac{c}{d} y_0 $ \\

  Then plug in these values of $x_1$ and $y_1$ to attain solutions: \\
  $ x = \frac{c}{d} x_0 + \frac{b}{d} t $
  $ y = \frac{c}{d} y_0 - \frac{a}{d} t $

  As an aside, note the negative in y's value. This seems to relate to the Euclidean algorithm.


  \pagebreak
  \subsection{Congruences}

  Rules for congruences:
  \begin{itemize}
    \item $\congr{a}{b}{n}$ and $\congr{b}{c}{n}$, then $\congr{a}{c}{n}$

    \item
      $\congr{a}{b}{n}$, then $\forall c \in \mathbb{Z}$
      \begin{itemize}
        \item $\congr{a \pm c}{b \pm c}{n}$
        \item $\congr{a c}{b c}{n}$
      \end{itemize}

    \item
      If $\congr{a_1}{b_1}{n}$ and $\congr{a_2}{b_2}{n}$, then:
      \begin{itemize}
        \item $\congr{a_1 \pm a_2}{b_1 \pm b_2}{n}$
        \item $\congr{a_1 a_2}{b_1 b_2}{n}$
      \end{itemize}

    \item
      $\forall\ 0 < c \in \mathbb{Z}$, the following are equivalent:
      \begin{itemize}
        \item $\congr{a}{b}{n}$
        \item $\congr{a}{b + n}{cn}$
        \item $\congr{a}{b + 2n}{cn}$
        \item \dots
        \item $\congr{a}{b + (c-1)n}{cn}$
      \end{itemize}

    \item
      If $\congr{ab}{ac}{n}$, then:
      \begin{itemize}
        \item $\congr{b}{c}{n}$ if $d\defeq$ gcd($a$, $n$) $= 1$
        \item $\congr{b}{c}{\frac{n}{d}}$ if $d\defeq$ gcd($a$, $n$) $> 1$
      \end{itemize}

    \item
      The linear congruence equation $\congr{ax}{b}{n}$ has a solution iff gcd($a$, $n$) $ = d | b$ and:
      \begin{itemize}
        \item if $d=1$, then the solution is unique mod $n$
        \item if $d>1$, then the solution is unique mod $\frac{n}{d}$
      \end{itemize}
  \end{itemize}

  \subsection{Congruence Equation $\congr{ax}{b}{n}$}

  Note that Linear Congruence Equations are \emph{equivalent} to
  Simple, Linear Diophantine Equations.

  Solving these requires applying rules 1-7 of the previous section. They are pretty arduous.

  \pagebreak
  \section{Abstract Algebra}

  For a group $\mathbf{G}$ the following must be true:
  \begin{description}
    \item[closure]
      $\forall g, h \in \mathbf{G}, g \cdot h \in \mathbf{G}$
    \item[associate - semigroup]
      $\forall a, b, c \in \mathbf{G}, a \cdot (b \cdot c) = (a \cdot b) \cdot c$
    \item[identity - monoid]
      $\exists e \in \mathbf{G} : \forall g \in \mathbf{G}, g \cdot e = e \cdot g = g$
    \item[inverse - group]
      $\forall g \in \mathbf{G}, \exists h \in \mathbf{G} : g \cdot h = h \cdot g = e$
  \end{description}

  Note that having closure is necessary. Adding associativity makes it a
  \emph{semigroup}. Adding an identity makes it a monoid. Adding an inverse,
  finally, makes a binary operator a \emph{\bf group}.

  \subsection{Cyclic Groups}

  Cyclic groups are defined by a generator, s.t.: \\
  $\mathbf{G} = \gen{a} = \set{a^n}{n \in \mathbb{Z}}$

  Furthermore, $\mathbb{Z}_n$ forms a group under modulo-addition. And the following is true:\\
  $m$ is a generator iff $m$ is coprime with $n$, the order of $\mathbb{Z}_n$

  \subsection{Subgroups}

  Let $(\mathbf{G},\cdot)$ be a group. If there exists $\mathbf{H} \subset
  \mathbf{G}$ s.t.  $(H,\cdot)$ is also a group, then we call
  $(\mathbf{H},\cdot)$ a subgroup of $\mathbf{G}$.

  Notably, we define the center of a group to be: \\
  $Z(\mathbf{G}) \defeq \set{z \in \mathbf{G}}{zg = gz \ \forall g \in \mathbf{G}}$

  And know that this is \emph{always} a subgroup (not merely a subset).

  \pagebreak
  \subsection{Cyclic Subgroups}

  Let $a \in \mathbf{G}$. With $(G,\cdot)$ being a group.\\
  $\gen{a} = \set{a^n}{n \in \mathbb{Z}}$

  $|\gen{a}|$ is the order of $<a>$, the lowest possible number, if finite, st\\
  $a^n = e$

  \begin{itemize}
    \item
      If $a \in \mathbf{G}$, then the cyclic subgroup $\gen{a}$ is the
      \emph{smallest} subgroup G containing $a$ 
    \item
      If $\mathbf{I}$ is an indexing set of $\mathbf{G}$, then $\gen{\{a_i\}}$ is the
      subgroup containing all finite products of $a_i \in \mathbf{G}$
  \end{itemize}

  \subsection{Generators and Relations}

  \DEFINITION{1} A \emph{\bf presentation} is a \emph{generating set} and a set
  of \emph{relations}.

  A \emph{presentation} can define a group. For example, Klein Group is defined by:\\
    generating set: $\{a, b\}$; relations: $a^2 = e, b^2 = e, ab = ba$

  The integers $\mathbb{Z} \modu{6}$ can be described:\\
    $\mathbb{Z}_6 \defeq $ generating set: $\{a\}$, relations: $a^6=e$

  \pagebreak
  \subsection{Some Theorems Concerning Subgroups}

  \begin{itemize}
    \item
      Let $\mathbf{G}$ be a finite, \emph{Abelian} group. If $\mathbf{H}$ is a
      subgroup of $\mathbf{G}$, the $|\mathbf{H}|$ divides $|\mathbf{G}|$
      ({\bf Lagrange's Theorem})

    \item
      Let $\mathbf{G}$ be a finite, abelian group of order $n$.
      Then $\mathbf{G}$ has at least one subgroup of order $d | n$, for every
      positive divisor $d$ of $n$. 

    \item
      \emph{If $\mathbf{G}$ is cyclic, we can do better.}\\
      Let $\mathbf{G}$ be a finite, cyclic group of order $n$. Then $\mathbf{G}$
      has \emph{exactly one} subgroup - a cyclic group at that - foreach
      positive divisor $d$ of $n$. If $\mathbf{G}$ is generated by $\gen{a}$,
      then $b = a^m$ has order $d = \frac{n}{\gcd(m, n)}$.
      Note that, $\gcd(n, 0) = n$

    \item
      Let $\mathbf{G}$ be a finite group of order $n$, and let $p$ be a prime
      that divides $n$ (that is, $p | n$).  Then, $\mathbf{G}$ has at least one
      subgroup of order $p$. ({\bf Cauchy's Theorem})

    \item
      Let $\mathbf{G}$ be a finite subgroup of order $n$. Let $n = p^k m$, where
      $p$ is prime and $p\not | m$. Then $\mathbf{G}$ has at least one subgroup
      of order $p^i \ \forall i \in [0 \dots k]$. ({\bf Sylow's first theorem})
  \end{itemize}


  \subsection{Isomorphism}

  Structural properties of groups:
  \begin{itemize}
    \item Order
    \item Abelian-ness
    \item Cyclic
    \item \emph{et cetera}
  \end{itemize}

  Note that, $\mathbf{G} \subset \mathbf{M}_2(\mathbb{Z}, \times) \cong$ Klein Group, when $\mathbf{G}$ is:\\
  $I = \begin{pmatrix}
    1 & 0 \\
    0 & 1
  \end{pmatrix}$
  $A = \begin{pmatrix}
    1 & 0 \\
    0 & -1
  \end{pmatrix}$, -I, -A


  Thus $\mathbf{M}_2(\mathbb{Z}) \cong \mathbf{V}_4$.


  \subsection{Classification of Finite Abelian Groups}

  First, let $\mathbf{G}_1$, $\mathbf{G}_2$ be finite Abelian on the set:\\
  $\mathbf{G}_1 \times \mathbf{G}_2 = \set{(a, b)}{a \in \mathbf{G}_1, b \in \mathbf{G}_2}$

  Define a binary operations, *, by the equation:\\
    $(a_1, a_2) * (b_1, b_2) = (a_1 *_1 b_1,\ a_2 *_2 b_2)$

  Some facts:
  \begin{itemize}
    \item
      The direct sum $\mathbb{Z}_m \times \mathbb{Z}_n$ is cyclic iff
      $\gcd(m, n) = 1$. If this is true, then
      $|\mathbb{Z}_m \times \mathbb{Z}_n| = m n$, and is isomorphic to $\mathbb{Z}_{mn}$.

    \item 
      The above can be generalized:
      $\mathbb{Z}_{m_1} \times \mathbb{Z}_{m_2} \times \dotso \times \mathbb{Z}_{m_k}$
      is cyclic iff $\gcd(m_i, m_j) = 1$ for every distinct pair. If this is
      true, then:
      $\mathbb{Z}_{m_1} \times \dots \times \mathbb{Z}_{m_k} \cong \mathbb{Z}_{m_1 \dotsm m_k}$

    \item 
      Every finite Abelian group $\mathbf{G}$ is isomorphic to \emph{some}
      direct sum of the form: \\
      $\mathbb{Z}_{(p_1)^{k_1}} \times \mathbb{Z}_{(p_2)^{k_2}} \times \dotsm \times \mathbb{Z}_{(p_n)^{k_n}}$ \\
      where $p_i$ are not necessarily distinct, and $k_i$ are not necessarily
      distinct. These are the \emph{elementary divisors} of $\mathbf{G}$.

    \item
      Alternatively, we can write it another way:\\
      $\mathbb{Z}_{m_1} \times \mathbb{Z}_{m_2} \times \dotsm \times \mathbb{Z}_{m_n}$\\
      Where $m_1 \geq 2$, $m_1 | m_2$, \dotso, $m_{n-1} | m_n$ \\
      Note that, $m_1, \dots, m_n$ are not distinct, but is a \emph{unique} list.
      These are the \emph{invariant factors} of the group.
  \end{itemize}

  \subsubsection{An Example}

  Show how many distinct (up to isomorphism) Abelian groups of order 600 exist:

  $600 = 2^3 \cdot 3 \cdot 5^2$ \\
  \begin{align*}
    600 & = 2 \cdot 2 \cdot 2\cdot 3 \cdot 5 \cdot 5  \\
        & = 2 \cdot 2^2 \cdot 3 \cdot 5 \cdot 5       \\
        & = 2^3 \cdot 3 \cdot 5 \cdot 5               \\
        & = 2 \cdot 2 \cdot 2\cdot 3 \cdot 5^2        \\
        & = 2 \cdot 2^2 \cdot 3 \cdot 5^2             \\
        & = 2^3 \cdot 3 \cdot 5^2                     \\
  \end{align*}

  Or rather $3 \cdot 1 \cdot 2 = 6$

  %
  %
  %
  \pagebreak
  \subsection{Group Homomorphisms}
  \label{sub-group-homomorphisms}

  $\phi$ is a homomorphism if: \\
  $\phi(a \cdot b) = \phi(a)\ast\phi(b)$

  \subsubsection{Properties of Homomorphisms}

  \begin{itemize}
    \item If $e$ is the identity element of $\mathbf{G}$, then $\phi(e)$ is the identity element of $\mathbf{G}'$
    \item If $g \in \mathbf{G}$ has finite order $m$, then $\phi(g)$ has order $m$
    \item If $a^{-1}$ is the inverse of $a \in \mathbf{G}$, then $\phi{a^{-1}}$ is the inverse of $\phi{a}\in\mathbf{G}'$
    \item If $\mathbf{H}$ is a subgroup of $\mathbf{G}$, then $\phi(\mathbf{H})$ is a subgroupf of $\mathbf{G}'$
    \item If $\mathbf{G}$ is finite, then the order of $\phi(\mathbf{G})$ divides the order of $\mathbf{G}$
    \item If $\mathbf{G}'$ is finite, then the order of $\phi(\mathbf{G})$ divides the order of $\mathbf{G}'$
    \item
      If $\mathbf{H}'$ is a subgroup of $\mathbf{G}'$, then $\phi^{-1}(\mathbf{H}')$ is a subgroup of $\mathbf{G}$, where: \\
      $\phi^{-1}(\mathbf{H}') = \set{h \in \mathbf{G}'}{\phi(h) \in \mathbf{H}'}$
  \end{itemize}

  {\noindent}The kernel of $\phi$ is defined:\\
  {\indent}$\ker \phi \defeq \set{g\in\mathbf{G}}{\phi(g) = e'}$
  And is a group, naturally.
   

  \subsection{Rings}


\end{document}
